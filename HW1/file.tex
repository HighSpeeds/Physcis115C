\documentclass[11pt]{article}
\author{Lawrence Liu}
\usepackage{subcaption}
\usepackage{graphicx}
\usepackage{amsmath,amssymb,stmaryrd}
\usepackage{physics}
\usepackage{pdfpages}
\newcommand{\Laplace}{\mathscr{L}}
\setlength{\parskip}{\baselineskip}%
\setlength{\parindent}{0pt}%
\usepackage{xcolor}
\usepackage{listings}
\definecolor{backcolour}{rgb}{0.95,0.95,0.92}
\usepackage{amssymb}
\lstdefinestyle{mystyle}{
    backgroundcolor=\color{backcolour}}
\lstset{style=mystyle}
\title{Physics 115C HW1}
\begin{document}
\maketitle
\section{Problem 1}
\subsection*{(a)}
We have that 
$$\vec{v_1}\vec{v_1}^T=
\begin{bmatrix}
    1 & 0 & 0 \\
    0 & 0 & 0 \\
    0 & 0 & 0
\end{bmatrix}$$
And:
$$\vec{v_2}\vec{v_2}^T=
\begin{bmatrix}
    0 & 0 & 0 \\
    0 & 1 & 0 \\
    0 & 0 & 0
\end{bmatrix}$$
And:
$$\vec{v_3}\vec{v_3}^T=
\begin{bmatrix}
    0 & 0 & 0 \\
    0 & 0 & 0 \\
    0 & 0 & 1
\end{bmatrix}$$
Therefore we have that 
$$\sum_{n} \vec{v_n}\vec{v_n}^T= \begin{bmatrix}
    1 & 0 & 0 \\
    0 & 1 & 0 \\
    0 & 0 & 1
\end{bmatrix}=\mathbf{1}_3$$
\subsection*{(b)}
We have that $\ket{\uparrow_x}=\frac{1}{\sqrt{2}}
\begin{bmatrix}
    1 \\
    1
\end{bmatrix}$ and $\ket{\downarrow_x}=\frac{1}{\sqrt{2}}
\begin{bmatrix}
    1 \\
    -1
\end{bmatrix}$, therefore we have that 
$$\ket{\uparrow_x}\bra{\uparrow_x}=\frac{1}{2}
\begin{bmatrix}
    1 & 1 \\
    1 & 1
\end{bmatrix}$$
And:
$$\ket{\downarrow_x}\bra{\downarrow_x}=\frac{1}{2}
\begin{bmatrix}
    1 & -1 \\
    -1 & 1
\end{bmatrix}$$
Therefore we have that
$$\sum_{n=\uparrow_x,\downarrow_x} \ket{n}\bra{n}=
\begin{bmatrix}
    1 & 0 \\
    0 & 1
\end{bmatrix}=\mathbf{1}_2$$
\subsection*{(c)}
We have that $\ket{\uparrow_y}=\frac{1}{\sqrt{2}}
\begin{bmatrix}
    1 \\
    i
\end{bmatrix}$
and $\ket{\downarrow_y}=\frac{1}{\sqrt{2}}
\begin{bmatrix}
    1 \\
    -i
\end{bmatrix}$, therefore we have that
$$\ket{\uparrow_y}\bra{\uparrow_y}=\frac{1}{2}
\begin{bmatrix}
    1\\
    i
\end{bmatrix}
\begin{bmatrix}
    1 & -i
\end{bmatrix}=
\frac{1}{2}\begin{bmatrix}
    1 & -i \\
    i & 1
\end{bmatrix}$$
And:
$$\ket{\downarrow_y}\bra{\downarrow_y}=\frac{1}{2}
\begin{bmatrix}
    1\\
    -i
\end{bmatrix}
\begin{bmatrix}
    1 & i
\end{bmatrix}=
\frac{1}{2}\begin{bmatrix}
    1 & i \\
    -i & 1
\end{bmatrix}$$
Therefore we have that
$$\sum_{n=\uparrow_y,\downarrow_y} \ket{n}\bra{n}=
\begin{bmatrix}
    1 & 0 \\
    0 & 1
\end{bmatrix}=\mathbf{1}_2$$
\subsection*{(d)}
We have that 
$$\ket{\uparrow_z}=\frac{1}{\sqrt{2}}\ket{\uparrow_y}
+\frac{1}{\sqrt{2}}\ket{\downarrow_y}$$
Since $\ket{\uparrow_y}$ is 
orthogonal to $\ket{\downarrow_y}$, we have that 
$\bra{\uparrow_y}\ket{\downarrow_y}=\bra{\downarrow_y}\ket{\uparrow_y}=0$.
Therefore we have that
$$\left(
    \ket{\uparrow_y}\bra{\uparrow_y}
    +\ket{\downarrow_y}\bra{\downarrow_y}
\right)
\ket{\uparrow_z}=
\frac{1}{\sqrt{2}}\ket{\uparrow_y}+
\frac{1}{\sqrt{2}}\ket{\downarrow_y}$$
This state is a superposition of 
$\ket{\uparrow_y}$ and $\ket{\downarrow_y}$, with probability 
$\frac{1}{2}$ for each. 
\subsection*{(e)}
We have that for a $n$ euclidean vectors $\ket{v_1},\ket{v_2},\ldots,\ket{v_n}$,
that span a space, we have that any $\ket{\psi}$, can we written 
as:
$$\ket{\psi}=\sum_{i=1}^N \ket{v_i}\bra{v_i}\ket{\psi}$$
Since $\sum_{i=1}^N \ket{v_i}\bra{v_i}= \mathbf{1}_n$, therefore 
$$\ket{\psi}=\mathbf{1}_n \ket{\psi}=\sum_{i=1}^N \ket{v_i}\bra{v_i}\ket{\psi}$$
If we generalize this to an infinite dimension Hilbert space, 
with continous variable $x$, we have that
$$\mathbf{1}= \int_{-\infty}^{\infty} dx\ket{x}\bra{x}$$
Therefore we have that 
$$\ket{\psi}=\mathbf{1}\ket{\psi}=\int_{-\infty}^{\infty} dx\ket{x}\bra{x}\ket{\psi}$$
$$\ket{\psi}=\int_{-\infty}^{\infty} dx\ket{x}\psi(x)$$
And 
$$\ket{\phi} = \int_{-\infty}^{\infty} dx\ket{x}\phi(x)$$
And thus 
$$\bra{\psi}\ket{\phi}=\int_{-\infty}^{\infty} \psi^{*}(x)\phi(x)dx$$
\section*{Problem 2}
TODO
\section*{Problem 3}
\subsection*{(a)}
We have that 
$$a_{+}\ket{n} = \sqrt{n+1}\ket{n+1}$$
$$a_{-}\ket{n} = \sqrt{n}\ket{n-1}$$
Therefore 
$$\bra{m}a_{+}\ket{n}= \sqrt{n+1}\bra{m}\ket{n+1}=
\sqrt{n+1}\delta_{m,n+1}$$
And
$$\bra{m}a_{-}\ket{n}= \sqrt{n}\bra{m}\ket{n-1}=
\sqrt{n}\delta_{m,n-1}$$
Therefore 
$$\bra{m}a_{+}\ket{n}
=\begin{bmatrix}
    0 & 0 & 0 & 0 & 0 & \cdots \\
    \sqrt{1} & 0 & 0 & 0 & 0 & \cdots \\
    0 & \sqrt{2} & 0 & 0 & 0 & \cdots \\
    0 & 0 & \sqrt{3} & 0 & 0 & \cdots \\
    0 & 0 & 0 & \sqrt{4} & 0 & \cdots \\
    \vdots & \vdots & \vdots & \vdots & \vdots & \ddots
\end{bmatrix}$$
and:
$$\bra{m}a_{-}\ket{n}
=\begin{bmatrix}
    0 & \sqrt{1} & 0 & 0 & 0 & \cdots \\
    0 & 0 & \sqrt{2} & 0 & 0 & \cdots \\
    0 & 0 & 0 & \sqrt{3} & 0 & \cdots \\
    0 & 0 & 0 & 0 & \sqrt{4} & \cdots \\
    0 & 0 & 0 & 0 & 0 & \cdots \\
    \vdots & \vdots & \vdots & \vdots & \vdots & \ddots
\end{bmatrix}$$
\subsection*{(b)}
We have that 
$$\hat{x}=\sqrt{\frac{\hbar}{2m\omega}}\left(
    a_{+}+a_{-}
\right)$$
$$\hat{p}=i\sqrt{\frac{\hbar m\omega}{2}}\left(
    a_{+}-a_{-}
\right)$$
Therefore we have that 
$$
\bra{m}\hat{x}\ket{n}=
\sqrt{\frac{\hbar}{2m\omega}}\left(
    \sqrt{n+1}\delta_{m,n+1}+\sqrt{n}\delta_{m,n-1}
\right)$$
$$\bra{m}\hat{x}\ket{n}=\sqrt{\frac{\hbar}{2m\omega}}
\begin{bmatrix}
    0 & \sqrt{1} & 0 & 0 & 0 & \cdots \\
    \sqrt{1} & 0 & \sqrt{2} & 0 & 0 & \cdots \\
    0 & \sqrt{2} & 0 & \sqrt{3} & 0 & \cdots \\
    0 & 0 & \sqrt{3} & 0 & \sqrt{4} & \cdots \\
    0 & 0 & 0 & \sqrt{4} & 0 & \cdots \\
    \vdots & \vdots & \vdots & \vdots & \vdots & \ddots
\end{bmatrix}$$
And:
$$
\bra{m}\hat{p}\ket{n}=
i\sqrt{\frac{\hbar m\omega}{2}}\left(
    \sqrt{n+1}\delta_{m,n+1}-\sqrt{n}\delta_{m,n-1}
\right)$$
$$
\bra{m}\hat{p}\ket{n}=
i\sqrt{\frac{\hbar m\omega}{2}}
\begin{bmatrix}
    0 & -\sqrt{1} & 0 & 0 & 0 & \cdots \\
    \sqrt{1} & 0 & -\sqrt{2} & 0 & 0 & \cdots \\
    0 & \sqrt{2} & 0 & -\sqrt{3} & 0 & \cdots \\
    0 & 0 & \sqrt{3} & 0 & -\sqrt{4} & \cdots \\
    0 & 0 & 0 & \sqrt{4} & 0 & \cdots \\
    \vdots & \vdots & \vdots & \vdots & \vdots & \ddots
\end{bmatrix}$$
These matrices are hermitian since $\bra{m}\hat{x}\ket{n}$ is real
and symmetric, and 
since $\bra{m}\hat{p}\ket{n}$ is purely imaginary and
skew-symmetric.
\subsection*{(c)}
We have that 
$$\hat{x}^2=\frac{\hbar}{2m\omega}\left(
    a_{+}^2+a_{-}^2+a_{+}a_{-}+ a_{-}a_{+}
\right)$$
Thus:
$$\hat{x}^2\ket{n}=
\frac{\hbar}{2m\omega}\left(
    \sqrt{(n+1)(n+2)}\ket{n+2}+\sqrt{n(n-1)}\ket{n-2}
    +(n+1)\ket{n}+n\ket{n}
\right)$$
Therefore 
$$\bra{m}\hat{x}^2\ket{n}=
\frac{\hbar}{2m\omega}\left(
    \sqrt{(n+1)(n+2)}\delta_{m,n+2}+\sqrt{n(n-1)}\delta_{m,n-2}
    +(n+1)\delta_{m,n}+n\delta_{m,n}
\right)$$
$$\bra{m}\hat{x}^2\ket{n}=
\frac{\hbar}{2m\omega}
\begin{bmatrix}
    1 & 0 & \sqrt{2} & 0 & 0 & \cdots \\
    0 & 3 & 0 & \sqrt{6} & 0 & \cdots \\
    \sqrt{2} & 0 & 5 & 0 & \sqrt{10} & \cdots \\
    0 & \sqrt{6} & 0 & 7 & 0 & \cdots \\
    0 & 0 & \sqrt{10} & 0 & 9 & \cdots \\
    \vdots & \vdots & \vdots & \vdots & \vdots & \ddots
\end{bmatrix}
$$
Likewise we have that:
$$\hat{p}^2=
\frac{\hbar m\omega}{2}\left(
    a_{+}^2+a_{-}^2-a_{+}a_{-}- a_{-}a_{+}  
\right)$$
$$\hat{p}^2\ket{n}=
\frac{\hbar m\omega}{2}\left(
    \sqrt{(n+1)(n+2)}\ket{n+2}-\sqrt{n(n-1)}\ket{n-2}
    -(n+1)\ket{n}-n\ket{n}  
\right)$$
$$\bra{m}\hat{p}^2\ket{n}=
-\frac{\hbar m\omega}{2}\left(
    \sqrt{(n+1)(n+2)}\delta_{m,n+2}-\sqrt{n(n-1)}\delta_{m,n-2}
    -(n+1)\delta_{m,n}-n\delta_{m,n}
\right)$$
$$\bra{m}\hat{p}^2\ket{n}=
-\frac{\hbar m\omega}{2}
\begin{bmatrix}
    -1 & 0 & \sqrt{2} & 0 & 0 & \cdots \\
    0 & -3 & 0 & \sqrt{6} & 0 & \cdots \\
    \sqrt{2} & 0 & -5 & 0 & \sqrt{10} & \cdots \\
    0 & \sqrt{6} & 0 & -7 & 0 & \cdots \\
    0 & 0 & \sqrt{10} & 0 & -9 & \cdots \\
    \vdots & \vdots & \vdots & \vdots & \vdots & \ddots
\end{bmatrix}$$
These matrices are hermitian since they are 
real and symmetric.
\subsection*{(d)}
We have that 
$$H=\hbar\omega \left(
    a_{+}a_{-}+\frac{1}{2}
\right)$$
Therefore we have that 
$$\bra{m}\hat{H}\ket{n}=
\hbar\omega\left(
    \bra{m}a_{+}a_{-}\ket{n}+\frac{1}{2}\delta_{m,n}\right)
    $$
From part (a) we have that:
$$\bra{m}a_{+}a_{-}\ket{n}=\begin{bmatrix}
    0 & 0 & 0 & 0 & 0 & \cdots \\
    \sqrt{1} & 0 & 0 & 0 & 0 & \cdots \\
    0 & \sqrt{2} & 0 & 0 & 0 & \cdots \\
    0 & 0 & \sqrt{3} & 0 & 0 & \cdots \\
    0 & 0 & 0 & \sqrt{4} & 0 & \cdots \\
    \vdots & \vdots & \vdots & \vdots & \vdots & \ddots
\end{bmatrix}\begin{bmatrix}
    0 & \sqrt{1} & 0 & 0 & 0 & \cdots \\
    0 & 0 & \sqrt{2} & 0 & 0 & \cdots \\
    0 & 0 & 0 & \sqrt{3} & 0 & \cdots \\
    0 & 0 & 0 & 0 & \sqrt{4} & \cdots \\
    0 & 0 & 0 & 0 & 0 & \cdots \\
    \vdots & \vdots & \vdots & \vdots & \vdots & \ddots
\end{bmatrix}
$$
$$\bra{m}a_{+}a_{-}\ket{n}
=\begin{bmatrix}
    0 & 0 & 0 & 0 & 0 & \cdots \\
    0 & 1 & 0 & 0 & 0 & \cdots \\
    0 & 0 & 2 & 0 & 0 & \cdots \\
    0 & 0 & 0 & 3 & 0 & \cdots \\
    0 & 0 & 0 & 0 & 4 & \cdots \\
    \vdots & \vdots & \vdots & \vdots & \vdots & \ddots
\end{bmatrix}
$$
Therefore 
$$\bra{m}\hat{H}\ket{n}=
\hbar\omega
    \begin{bmatrix}
    \frac{1}{2} & 0 & 0 & 0 & 0 & \cdots \\
    0 & 1+\frac{1}{2} & 0 & 0 & 0 & \cdots \\
    0 & 0 & 2+\frac{1}{2} & 0 & 0 & \cdots \\
    0 & 0 & 0 & 3+\frac{1}{2} & 0 & \cdots \\
    0 & 0 & 0 & 0 & 4+\frac{1}{2} & \cdots \\
    \vdots & \vdots & \vdots & \vdots & \vdots & \ddots
\end{bmatrix}
$$
\subsection*{(e)}
We have that
$$\hat{H}=\frac{\hat{p}^2}{2m}+\frac{1}{2}
m\omega^2\hat{x}^2$$
Therefore we have that from part (c):
$$\bra{m}\hat{H}\ket{n}=
\frac{1}{2m}\bar{m}\hat{p}^2\ket{n}+
\frac{1}{2}m\omega^2\bar{n}\hat{x}^2\ket{n}$$
$$\bra{m}\hat{H}\ket{n}=-\frac{\omega\hbar}{4}
\begin{bmatrix}
    -1 & 0 & \sqrt{2} & 0 & 0 & \cdots \\
    0 & -3 & 0 & \sqrt{6} & 0 & \cdots \\
    \sqrt{2} & 0 & -5 & 0 & \sqrt{10} & \cdots \\
    0 & \sqrt{6} & 0 & -7 & 0 & \cdots \\
    0 & 0 & \sqrt{10} & 0 & -9 & \cdots \\
    \vdots & \vdots & \vdots & \vdots & \vdots & \ddots
\end{bmatrix}
+\frac{1}{2}m\omega^2\frac{\hbar}{2m\omega}
\begin{bmatrix}
    1 & 0 & \sqrt{2} & 0 & 0 & \cdots \\
    0 & 3 & 0 & \sqrt{6} & 0 & \cdots \\
    \sqrt{2} & 0 & 5 & 0 & \sqrt{10} & \cdots \\
    0 & \sqrt{6} & 0 & 7 & 0 & \cdots \\
    0 & 0 & \sqrt{10} & 0 & 9 & \cdots \\
    \vdots & \vdots & \vdots & \vdots & \vdots & \ddots
\end{bmatrix}
$$
Thus we get 
$$
\bra{m}\hat{H}\ket{n}
=\frac{\omega\hbar}{4}
\begin{bmatrix}
    2 & 0 & 0 & 0 & 0 & \cdots \\
    0 & 6 & 0 & 0 & 0 & \cdots \\
    0 & 0 & 10 & 0 & 0 & \cdots \\
    0 & 0 & 0 & 14 & 0 & \cdots \\
    0 & 0 & 0 & 0 & 18 & \cdots \\
    \vdots & \vdots & \vdots & \vdots & \vdots & \ddots
\end{bmatrix}$$
Which is the same matrix we found in part (d). Since this 
matrix only has nonzero elements on the diagonal and the
values are real, this matrix is hermitian.
\subsection*{(f)}
We have that 
$$a_{+}\ket{2}=\sqrt{3}\ket{3}$$
$$a_{-}\ket{2}=\sqrt{2}\ket{1}$$
Therefore we have that 
$$\hat{x}\ket{2}=\sqrt{\frac{\hbar}{2m\omega}}(\sqrt{3}\ket{3}+\sqrt{2}\ket{1})$$
$$\hat{p}\ket{2}=i\sqrt{\frac{\hbar m\omega}{2}}(\sqrt{3}\ket{3}-\sqrt{2}\ket{1})$$
And:
$$\hat{x}^2\ket{2}=\frac{\hbar}{2m\omega}(\sqrt{12}\ket{4}
+\sqrt{2}\ket{0}+5\ket{2})$$
$$\hat{p}^2\ket{2}=-\frac{\hbar m\omega}{2}(\sqrt{12}\ket{4}
+\sqrt{2}\ket{0}-5\ket{2})$$
Therefore we have that 
$$\hat{H}\ket{2}=
\hbar\omega\left(2+\frac{1}{2}\right)\ket{2}$$
\section*{Problem 4}
Let us consider the one dimensional case first. If a state has definite parity then 
we have that $|\bra{x}\ket{n}|^2$ is even. Therefore we have that 
$$\bra{n}x\ket{n}=\int x|\bra{x}\ket{n}|^2 dx=0$$
Therefore the dipole moment of a stationary state is $0$ in one dimension. 
Generalizing to 3 dimensions we have that $\bra{r}\ket{n}$ is even along all 
3 dimensions, ie that $\bra{r}\ket{n}=\psi(x,y,z)=
\psi(-x,y,z)=\psi(x,-y,z)=\psi(x,y,-z)$. Therefore we have that 
$$\bra{n}r\ket{n}=\int r|\bra{r}\ket{n}|^2 d^3r=0$$
Therefore we have that
$$\bra{n}(qr)\ket{n}=\bra{n}\mathbf{p}\ket{n}=0$$




\end{document}
\documentclass[11pt]{article}
\author{Lawrence Liu}
\usepackage{subcaption}
\usepackage{graphicx}
\usepackage{amsmath,amssymb,stmaryrd}
\usepackage{physics}
\usepackage{pdfpages}
\newcommand{\Laplace}{\mathscr{L}}
\setlength{\parskip}{\baselineskip}%
\setlength{\parindent}{0pt}%
\usepackage{xcolor}
\usepackage{listings}
\definecolor{backcolour}{rgb}{0.95,0.95,0.92}
\usepackage{amssymb}
\lstdefinestyle{mystyle}{
    backgroundcolor=\color{backcolour}}
\lstset{style=mystyle}
\title{Physics 115C HW 8}
\begin{document}
\maketitle
\section*{Problem 1}
\subsection*{(a)}
We have that the beam splitter matrix is:
$$
\frac{1}{\sqrt{2}}
\begin{bmatrix}
    1 & 1\\
    1 & -1
\end{bmatrix}$$
We note that the beam splitter matrix is unitary, therefore we have that
$$\ket{i_1} = \frac{1}{\sqrt{2}}\left(\ket{O_1}-\ket{O_2}\right)$$
$$\ket{i_2} = \frac{1}{\sqrt{2}}\left(\ket{O_1}+\ket{O_2}\right)$$
We also have that the state after the beamsplitter of a photon in state is 
a superposition of the two output states ie we have that the output state is:
$$\frac{1}{\sqrt{2}}\left(\ket{O_1}+\ket{O_2}\right)$$
\subsection*{(b)}
We have that 
$$\ket{D_1} = -\frac{1}{\sqrt{2}}\left(\ket{O_1}+\ket{O_2}\right)$$
$$\ket{D_2} = -\frac{1}{\sqrt{2}}\left(\ket{O_2}-\ket{O_1}\right)$$
Thus we have 
$$\ket{D_1} = -\frac{1}{2}\left(\ket{i}-\ket{i}\right)$$
$$\ket{D_2} = -\frac{1}{2}\left(\ket{i}+\ket{i}\right)$$
\subsection*{(c)}
Thus we have that 
$$|bra{i}\ket{D_1}|^2 = 0$$
and 
$$|bra{i}\ket{D_2}|^2 = 1$$
Thus we can see that all the photons are detected by detector 2.
\subsection*{(d)}
We have in this case:
$$\ket{D_1} = -\frac{1}{2}\left(\ket{i}+\ket{i}\right)$$
$$\ket{D_2} = -\frac{1}{2}\left(\ket{i}-\ket{i}\right)$$
Thus we have that
$$|\bra{i}\ket{D_1}|^2 = 1$$
and
$$|\bra{i}\ket{D_2}|^2 = 0$$
Thus we can see that all the photons are detected by detector 1.
\section*{Problem 2}
\subsection*{(a)}
For each state, we can decompose $r$ into $x$, $y$, and $z$. And there 
are 4 states we are intrested in, so we need to calculate 12 matrix elements. 
\subsection*{(b)}
We have that from hw 6:
$$\bra{100}z\ket{200} =0$$
$$\bra{100}z\ket{211} =0$$
$$\bra{100}z\ket{21-1} =0$$
Likewise we have that 
$$\bra{100}x\ket{200} =0$$
$$\bra{100}x\ket{210} = 0$$
and 
$$\bra{100}y\ket{200} =0$$
$$\bra{100}y\ket{210} = 0$$
\subsection*{(c)}
We have that from HW 6:
$$\bra{100}z\ket{210} = \frac{2^{\frac{15}{2}}a_0}{3^5}$$
And:
\begin{align*}
    \bra{100}x\ket{21\pm1} &= \int_{0}^{2\pi}\int_{0}^{\pi}\int_{0}^{\infty} \phi_{100}^*x\phi_{21\pm1}r^2\sin(\theta)drd\theta d\phi\\
    &= \int_{0}^{2\pi}\int_{0}^{\pi}\int_{0}^{\infty} \phi_{100}^*\phi_{21\pm1} r^3\sin^2(\theta)\cos(\phi)drd\theta d\phi\\
    &= \frac{1}{8a_0^4\pi }\int_{0}^{2\pi}\cos(\phi)e^{\pm i\phi}\int_{0}^{\pi}\sin^3(\theta)\int_{0}^{\infty} r^4e^{-3r/(2a_0)}drd\theta d\phi\\
    &=\frac{1}{8a_0^4\pi}\frac{256a_0^5}{81}\frac{4}{3} \pi\\
    &= \frac{128}{243}a_0
\end{align*}
Likewise we have that:
\begin{align*}
    \bra{100}y\ket{21\pm1} &= \int_{0}^{2\pi}\int_{0}^{\pi}\int_{0}^{\infty} \phi_{100}^*y\phi_{21\pm1}r^2\sin(\theta)drd\theta d\phi\\
    &= \int_{0}^{2\pi}\int_{0}^{\pi}\int_{0}^{\infty} \phi_{100}^*\phi_{21\pm1} r^3\sin^2(\theta)\sin(\phi)drd\theta d\phi\\
    &= \frac{1}{8a_0^4\pi \sqrt{3}}\int_{0}^{2\pi}\sin(\phi)e^{\pm i\phi}\int_{0}^{\pi}\sin^3(\theta)\int_{0}^{\infty} r^4e^{-3r/(2a_0)}drd\theta d\phi\\
    &=\pm\frac{1}{8a_0^4\pi \sqrt{3}}\frac{256a_0^5}{81}\frac{4}{3} i\pi
    &= \pm\frac{i128}{243}a_0
\end{align*}
\subsection*{(d)}
We have that the transistion rates are:
$$\Gamma_{210\to 100} = \frac{e^2\omega^3}{3\pi\epsilon_0\hbar c^3}\left|\bra{100}z\ket{210}\right|^2 = 6.260\times 10^{8} s^{-1}$$
$$\Gamma_{211\to 100} = \frac{e^2\omega^3}{3\pi\epsilon_0\hbar c^3}\left(\left|\bra{100}x\ket{211}\right|^2+\left|\bra{100}y\ket{211}\right|^2\right)= 6.260\times 10^{8} s^{-1}$$
$$\Gamma_{21-1 \to 100} = \frac{e^2\omega^3}{3\pi\epsilon_0\hbar c^3}\left(\left|\bra{100}x\ket{21-1}\right|^2+\left|\bra{100}y\ket{21-1}\right|^2\right)= 6.260\times 10^{8} s^{-1}$$
Therefore we get that lifetime for each are $1.597$ns. 
Likewise since $\bra{100}x\ket{200} = \bra{100}y\ket{200} = \bra{100}z\ket{200} = 0$ we have that
$$\Gamma_{200\to 100} = 0$$
Therefore the lifetime for this state is infinite.
\subsection*{(e)}
The values we calculated for the $\ket{210}$ and $\ket{21\pm}$ state is very close to what we theoretically calculated. However the 
the transition lifetime for the $\ket{200}$ state is not infinite, which is not what we theoretically calculated, however it is several
orders of magnitude larger than the other states, so it is still a very long lifetime. The reason why the transistion lifetime is not infinite
is because of intermediate states. 
\section*{Problem 3}
\subsection*{(a)}
We have 
\begin{align*}
    H  &= \frac{P^2}{2M}+\frac{p^2}{2m}+V(x)\\
    &= \frac{p_1^2+2p_1p_2+p_2^2}{2(m_1+m_2)}+\frac{m_2^2p_1-2m_1m_2p_1+m_1^2p_2}{2m_1m_2(m_1+m_2)}+V(x)\\
    &= \frac{m_1m_2p_1^2+m_1m_2p_2^2}{2m_1m_2(m_1+m_2)}+\frac{m_2^2p_1+m_1^2p_2}{2m_1m_2(m_1+m_2)}+V(x)\\
    &= \frac{m_2(m_1+m_2)p_2^2+m_1(m_1+m_2)p_1^2}{2m_1m_2(m_1+m_2)}+V(x)\\
    &= \frac{p_1^2}{2m_1}+\frac{p_2^2}{2m_2}+V(|x_1-x_2|)
\end{align*}
\subsection*{(b)}
We have that: 
$$[X_i,P_i] = \frac{m_1[p_{1i},x_{1i}]+m_2[p_{2i},x_{2i}]}{m_1+m_2} = i\hbar$$
and for $i\neq j$:
$$[X_i,P_j] = \frac{m_1[p_{1i},x_{1j}]+m_2[p_{2i},x_{2j}]}{m_1+m_2} = 0$$
Likewise:
$$[x_i,p_i] = \frac{m_2}{m_1+m_2}[x_{1i},p_{1i}]+\frac{m_1}{m_1+m_2}[x_{2i},p_{2i}] = i\hbar$$
and for $i\neq j$:
$$[x_i,p_j] = \frac{m_2}{m_1+m_2}[x_{1i},p_{1j}]+\frac{m_1}{m_1+m_2}[x_{2i},p_{2j}] = 0$$
\subsection*{(c)}
We have that thus the Schrodinger equation is:
$$\left(\frac{P^2}{2M}+\frac{p^2}{2m}+V(x)\right) \psi_{CM}(X)\psi_{rel}(x) = E\psi_{CM}(X)\psi_{rel}(x)$$
\subsection*{(d)}
Dividing by $\psi_{CM}(X)\psi_{rel}(x)$ we get:
$$\frac{1}{\psi_{CM}(X)}\frac{P^2}{2M}\psi_{CM}(X)+\frac{1}{\psi_{rel}(x)}\frac{p^2}{2m}\psi_{rel}(x)+\frac{1}{\psi_{rel}(x)}V(x)\psi_{rel}(x) = E$$
therefore we can see that we have a term $\frac{1}{\psi_{CM}(X)}\frac{P^2}{2M}\psi_{CM}(X)$ that only 
depends on $X$ and a term $\frac{1}{\psi_{rel}(x)}\frac{p^2}{2m}\psi_{rel}(x)+\frac{1}{\psi_{rel}(x)}V(x)\psi_{rel}(x)$ that only depends on $x$.
\subsection*{(e)}
If we have: 
$$\frac{P^2}{2M}\psi_{CM}(X) = E_{CM}\psi_{CM}(X)$$
and 
$$\frac{p^2}{2m}\psi_{rel}(x)+V(x)\psi_{rel}(x) = E_{rel}\psi_{rel}(x)$$
then we have that substituting back into the equation we derived above:
$$E_{CM}\frac{1}{\psi_{CM}(X)}\psi_{CM}(X)+E_{rel}\frac{1}{\psi_{rel}(x)}\psi_{rel}(x) = E_{CM}+E_{rel} = E$$
\subsection*{(f)}
By taking into account the mass of the proton and the electron we have the 
reduced mass is:
$$\mu = 9.104\times 10^{-28} kg$$
The mass of the proton far outweighs the 
mass of the electron. Therefore we can see that the relative momentum 
is approximately the momentum of the electron. 
\section*{Problem 4}
\subsection*{(a)}
We have 
$$\frac{1}{2}\left(\ket{r_1}\ket{r_2}+ket{r_1}\ket{t_2}+\ket{t_1}\ket{r_2}+\ket{t_1}\ket{t_2}\right)$$
\subsection*{(b)}
Therefore the probability that Detector 1 and Detector 2 both click is $0.5$. And the probability 
that Detector 1 clicks twice is $0.25$ and the probability that Detector 2 clicks twice is $0.25$.
\subsection*{(c)}
If the photons are indistringuishable then we have that the state is 
$$\frac{1}{2}\left(-\ket{D_2}\ket{D_1}-\ket{D_2}\ket{D_2}+\ket{D_1}\ket{D_2}+\ket{D_1}\ket{D_1}\right)$$
Then if renormalizing then we have:
$$\frac{1}{\sqrt{2}}\left(\ket{D_1}\ket{D_1}-\ket{D_2}\ket{D_2}\right)$$
\subsection*{(d)}
Therefore we can see that the probability that Detector 1 and Detector 2 both click is $0$ and the 
probability that Detector 1 clicks twice is $0.5$ and the probability that Detector 2 clicks twice is $0.5$.
\subsection*{(e)}
A single photon can interfere with other photon only if they are 
indistinguishable. If they are distinguishable then they will not interfere.
\subsection*{(f)}
We have that $P_C=0$ thus 
$$\alpha  = \frac{P_C}{P_{D1}P_{D2}} = 0$$

\end{document}




\documentclass[11pt]{article}
\author{Lawrence Liu}
\usepackage{subcaption}
\usepackage{graphicx}
\usepackage{amsmath,amssymb,stmaryrd}
\usepackage{physics}
\usepackage{pdfpages}
\newcommand{\Laplace}{\mathscr{L}}
\setlength{\parskip}{\baselineskip}%
\setlength{\parindent}{0pt}%
\usepackage{xcolor}
\usepackage{listings}
\definecolor{backcolour}{rgb}{0.95,0.95,0.92}
\usepackage{amssymb}
\lstdefinestyle{mystyle}{
    backgroundcolor=\color{backcolour}}
\lstset{style=mystyle}
\title{Physics 115C HW 3}
\begin{document}
\maketitle
\section*{Problem 1}
\subsection*{(a)}
It would take $(-13.6\text{eV} - (-14.36 \text{eV})) = 0.76\text{eV}$ to remove one of the 
two electrons from the Hydrogen anion.
\subsection*{(b)}
We would expect that the ground state energy of the Hydrogen anion without
electron electron repulsion would be $-27.2\text{eV}$. And thus it would take 
$(-13.6\text{eV} - (-27.2 \text{eV})) = 13.6\text{eV}$ to remove one of the
electrons.
\subsection*{(c)}
If the helium atom had 3 electrons, we would expect two to fill up the 
$1s$ orbital and one to fill up the $2s$ orbital. If we ignore
electron electron repulsion, we would expect that the 1s energy 
of the helium atom would be $-54.4\text{eV}$, since there is 
two electrons we would have $-108.8\text{eV}$, and the
2s electron would have an energy of $-13.6\text{eV}$. Therefore
we would expect the ground state energy to be 
$-122.4\text{eV}$.
\subsection*{(d)}
If there were two electrons, they would be more likely to be on the 
opposite sides of the nucleus. Thus the "shielding" effect of the 
other electron would be less. Where as from the pauli inclusion 
exclusion principle we know that for 3 electrons, one of them have 
to be in a different state. Therefore these electrons won't be on the opposite
side of the nucleus and thus the "shielding" effect of the other electrons will be higher.
\section*{Problem 2}
\subsection*{(a)}
We have that from the Virial theorem for a 
harmonic oscillator that
$$\langle T\rangle = \langle V\rangle$$
Therefore:
$$\langle T \rangle =  \frac{E_n}{2}$$
We have that our trial wavefunction is effectively the 
equal to the wavefunction for a harmonic oscillator with
$\omega=\lambda$. And since the energy of the ground state of a 
harmonic oscillator is given by
$$E_n = \hbar\omega\frac{1}{2}$$
We have that
$$\bra{\psi_{trial}}T\ket{\psi_{trial}} = \boxed{\frac{\hbar\lambda}{4}}$$.
\subsection*{(b)}
We shall prove that 
$$\int_{-\infty}^{\infty} x^{2n}e^{-ax^2}dx = \frac{\prod_{k=1}^n (2k-1)}{(2a)^n}\sqrt{\frac{\pi}{a}}$$
Through induction, for $n=1$ we have that 
$$\int_{-\infty}^{\infty} x^{2}e^{-ax^2}dx =\sqrt{\frac{\pi}{a}}\int_{-\infty}^{\infty} \sqrt{\frac{a}{\pi}}x^{2}e^{-\frac{1}{2}\frac{x^2}{\frac{1}{2a}}}dx $$
As we can see the integral is now the integral for the second moment of a gaussian with variance 
$\frac{1}{2a}$ centered around 0. Therefore we have that 
$$\int_{-\infty}^{\infty} x^{2}e^{-ax^2}dx =\frac{1}{2a}\sqrt{\frac{\pi}{a}}$$
Now we assume that the integral holds for $n=k$ and we shall prove that it holds for $n=k+1$.
We have that
\begin{align*}
    -\int_{-\infty}^{\infty} x^{2k+2}e^{-ax^2}dx &= \frac{d}{da}\int_{-\infty}^{\infty} x^{2k}e^{-ax^2}dx\\
    &= \frac{d}{da}\frac{\sqrt{\pi}\prod_{k=1}^n (2k-1)}{2^na^{n+\frac{1}{2}}}\\
    &= -\frac{\sqrt{\pi}(n+\frac{1}{2})\prod_{k=1}^n (2k-1)}{2^na^{n+\frac{1}{2}+1}}\\
    \int_{-\infty}^{\infty} x^{2k+2}e^{-ax^2}dx&= \boxed{\frac{\prod_{k=1}^{n+1} (2k-1)}{(2a)^{n+1}}\sqrt{\frac{\pi}{a}}}
\end{align*}
We have that 
\begin{align*}
    \bra{\psi_{trial}}V\ket{\psi_{trial}} &= \left(\frac{m\lambda}{\pi\hbar}\right)^\frac{1}{2} \int_{-\infty}^{\infty} kx^4 e^{-\frac{m\lambda}{\hbar}x^2}dx\\
    &= \boxed{\frac{3k}{2^2\left(\frac{m\lambda}{\hbar}\right)^2}}\\
\end{align*}
\subsection*{(c)}
We have that 
$$\bar{H} = \frac{\hbar \lambda}{4}+\frac{3k}{4\left(\frac{m\lambda}{\hbar}\right)^2}$$
Taking the derivative and setting it equal to 0 we have that:
\begin{align*}
    \frac{\hbar}{4} - \frac{2\cdot3k\hbar^2}{4m^2\lambda^3} &= 0\\
    \hbar &= \frac{2\cdot3k\hbar^2}{m^2\lambda^3}\\
    \lambda^3 &= \frac{2\cdot3k\hbar}{m^2}\\
    \lambda &= \boxed{\left(\frac{2\cdot3k\hbar}{m^2}\right)^\frac{1}{3}}
\end{align*}
\subsection*{(d)}
We thus have that for $\hbar = m = 1$ and $k=\frac{1}{2}$:
$$\lambda = 3^\frac{1}{3}$$
Therefore we have that 
$$\bar{H} =  \frac{3^\frac{1}{3}}{4}+\frac{3}{8\cdot3^\frac{2}{3}} = \boxed{0.540}$$
Which is $\boxed{2\%}$ away from the numerical value of $0.53$.
\section*{Problem 3}
\subsection*{(a)}
We have that 
$$
    \begin{pmatrix}
        \dot{c_1}\\
        \dot{c_2}\\
    \end{pmatrix}=
    \frac{1}{i\hbar}
    \begin{pmatrix}
        0 & \frac{\gamma}{2}e^{-i(\omega_{21}-\omega)t}\\
        \frac{\gamma}{2}e^{i(\omega_{21}-\omega)t} & 0\\
    \end{pmatrix}
    \begin{pmatrix}
        c_1\\
        c_2\\
    \end{pmatrix}
$$
When $\delta \omega = \omega_{21}-\omega$ we have that 
$$
    \begin{pmatrix}
        \dot{c_1}\\
        \dot{c_2}\\
    \end{pmatrix}=
    \frac{1}{i\hbar}
    \begin{pmatrix}
        0 & \frac{\gamma}{2}e^{-i\delta\omega t}\\
        \frac{\gamma}{2}e^{i\delta\omega t} & 0\\
    \end{pmatrix}
    \begin{pmatrix}
        c_1\\
        c_2\\
    \end{pmatrix}
$$
\subsection*{(b)}
We have that 
\begin{align*}
    \dot{c_2}&=\frac{1}{i\hbar}\frac{\gamma}{2}e^{i\delta\omega t}c_1\\
    \ddot{c_2}&=\frac{1}{i\hbar}\frac{\gamma}{2}e^{i\delta\omega t}\dot{c_1}+\frac{\delta\omega\gamma}{2\hbar}e^{i\delta\omega t}c_1\\
    &=-\frac{\gamma^2}{4\hbar^2}c_2+i\delta\omega \dot{c_2}\\
    \ddot{c_2}-i\delta\omega \dot{c_2}+\frac{\gamma^2}{4\hbar^2}c_2&=0
\end{align*}
\subsection*{(c)}
We have that the corresponding characteristic equation is
$$\lambda^2-i\delta\omega\lambda+\frac{\gamma^2}{4\hbar^2}=0$$
Therefore we have 
$$\lambda = \frac{i\delta\omega\pm\sqrt{-\delta\omega^2-\frac{\gamma^2}{\hbar^2}}}{2}$$
$$\lambda = i\frac{\delta\omega+\Omega}{2}$$
Where $\Omega = \sqrt{\delta\omega^2+\frac{\gamma^2}{\hbar^2}}$
Therefore we have that the solution is off the form 
$$c_2(t) = Ae^{i\frac{\delta\omega+\Omega}{2}t}+Be^{i\frac{\delta\omega-\Omega}{2}t}$$
Which we can rearage to:
$$c_2(t) = C_{+}e^{i\xi_+ t}+C_{-}e^{i\xi_- t}$$
Where $\xi_{\pm} = \frac{\delta\omega\pm\Omega}{2}$.
\subsection*{(d)}
We have that:
\begin{align*}
    c_2(0) &= 0\\
    C_{+}+C_{-} &= 0\\
    C_{-} &= -C_{+}\\
\end{align*}
Likewise we have that 
\begin{align*}
    c_1(0)  &= 1\\
    i\hbar\frac{2}{\gamma}\dot{c_2}(0) &= 1\\
    i\hbar\frac{2}{\gamma}\left(-i\xi_{+}C_{-}+i\xi_{-}C_{-}\right) &= 1\\
    \frac{2i\hbar}{\gamma}\left(\xi_{-}-\xi_{+}\right)C_{-} &= 1\\
    \frac{2i\hbar}{\gamma}\left(-\Omega\right)C_{-} &= 1\\
    C_{-} &= -\frac{\gamma}{2i\hbar\Omega}
\end{align*}
Therefore we have that
$$c_{2}(t) = \frac{\gamma}{2i\hbar\Omega}\left(e^{i\xi_{+}t}-e^{i\xi_{-}t}\right)$$
And thus we have that 
$$c_{2}^{*}(t) = \frac{-\gamma}{2i\hbar\Omega}\left(e^{-i\xi_{+}t}-e^{-i\xi_{-}t}\right)$$
\begin{align*}
    |c_{2}(t)|^2 &= \frac{\gamma^2}{4\hbar^2\Omega^2}\left(e^{i\xi_{+}t}-e^{i\xi_{-}t}\right)\left(e^{-i\xi_{+}t}-e^{-i\xi_{-}t}\right)\\
    &= \frac{\gamma^2}{4\hbar^2\Omega^2}\left(2-2\cos\left(\xi_{+}-\xi_{-}\right)t\right)\\
    &= \frac{\gamma^2}{2\hbar^2\Omega^2}\left(1-\cos\left(\Omega t\right)\right)\\
    &= \frac{\gamma^2}{\hbar^2\Omega^2}\sin^2\left(\frac{\Omega t}{2}\right)\\
\end{align*}
\subsection*{(e)}
The maximum value of $|c_2(t)|^2$ is $\frac{\gamma^2}{\hbar^2\Omega^2}$ which is reached when $\sin^2\left(\frac{\Omega t}{2}\right) = 1$.
We have that 
\begin{align*}
    \frac{\gamma^2}{\hbar^2\Omega^2} &= \frac{\gamma^2}{\hbar^2(\delta\omega^2+\frac{\gamma^2}{\hbar^2})}
\end{align*}
Therefore we can see that when $\delta\omega=0$, we recover back the result for the 
rabbi oscialltion at resonance, and for values of $\delta\omega\neq 0$ we have that 
$\delta\omega^2>0$ and thus $\frac{\gamma^2}{\hbar^2(\delta\omega^2+\frac{\gamma^2}{\hbar^2})}<1$.
\subsection*{(f)}
Since the particle can only take two possible states, 
we have that at equal superposition, we must have 
\begin{align*}
    |c_{2}(t)|^2 &= \frac{1}{2}\\
    \frac{\gamma^2}{\hbar^2\Omega^2}\sin^2\left(\frac{\Omega t}{2}\right) &= \frac{1}{2}\\
    \sin\left(\frac{\Omega t}{2}\right) &= \frac{\hbar \Omega}{\gamma\sqrt{2}}\\
    \frac{\Omega t}{2} &= \arcsin\left(\frac{\hbar \Omega}{\gamma\sqrt{2}}\right)\\
    t &= \frac{2}{\Omega}\arcsin\left(\frac{\hbar \Omega}{\gamma\sqrt{2}}\right)\\
\end{align*}
Thus we can see that the time it takes for the particle to reach equal superposition increases
as $|\delta\omega|$ increases and it takes longer than 
at resonance.
\section*{Problem 4}
\subsection*{(a)}
We have that 
$$H_0+H_1 = \begin{pmatrix}
    2\hbar\omega & \hbar\lambda\\
    \hbar\lambda & 0
\end{pmatrix}$$
In order for $E_{\pm}$ to be eigenvalue and $v_{\pm}$ to be 
eigenvectors, we must have that
\begin{align*}
    (H_0+H_1)v_{+} &= E_{+}v_{+}\\
    d \begin{pmatrix}
        2\hbar\omega(\omega+\Delta)+\hbar\lambda^2\\
        \hbar\lambda(\omega+\Delta)
    \end{pmatrix} &= \hbar(\omega+\Delta)d\begin{pmatrix}
        \omega+\Delta\\
        \lambda
    \end{pmatrix}\\
    \begin{pmatrix}
        2\hbar\omega^2+2\hbar\omega\Delta+\hbar\lambda^2\\
        \hbar\lambda(\omega+\Delta)
    \end{pmatrix} &= \begin{pmatrix}
        \hbar(\omega+\Delta)^2\\
        \hbar\lambda(\omega+\Delta)
    \end{pmatrix}\\
    \begin{pmatrix}
        \hbar\omega^2+2\hbar\omega\Delta+\hbar\Delta^2\\
        \hbar\lambda(\omega+\Delta)
    \end{pmatrix} &= \begin{pmatrix}
        \hbar\omega^2+2\hbar\omega\Delta+\hbar\Delta^2\\
        \hbar\lambda(\omega+\Delta)
    \end{pmatrix}
\end{align*}
\begin{align*}
    (H_0+H_1)v_{-} &= E_{-}v_{-}\\
    d \begin{pmatrix}
        -2\hbar\omega\lambda + \hbar\lambda(\omega+\Delta)\\
        -\hbar\lambda^2
    \end{pmatrix} &= \hbar(\omega-\Delta)d\begin{pmatrix}
        -\lambda\\
        \omega+\Delta
    \end{pmatrix}\\
    \begin{pmatrix}
        -\hbar\omega\lambda + \hbar\lambda\Delta\\
        -\hbar\lambda^2
    \end{pmatrix} &= \begin{pmatrix}
        -\hbar\omega\lambda + \hbar\lambda\Delta\\
        \hbar(\omega^2-\Delta^2)
    \end{pmatrix}\\
    \begin{pmatrix}
        -\hbar\omega\lambda + \hbar\lambda\Delta\\
        -\hbar\lambda^2
    \end{pmatrix} &=
    \begin{pmatrix}
        -\hbar\omega\lambda + \hbar\lambda\Delta\\
        -\hbar\lambda^2
    \end{pmatrix}
\end{align*}
Therefore we can see that $v_{+}$ and $v_{-}$ are the 
eigenvectors of $H_0+H_1$ and $E_{+}$ and $E_{-}$ are the
eigenvalues of $H_0+H_1$.
We can see that the magnitudes of the eigenvalues are
$1$ as well since $(\omega+\Delta)^2+\lambda^2=2\omega\Delta+\Delta^2+\omega^2+\lambda^2=2\Delta(\omega+\Delta)=d^{-2}$.
\subsection*{(b)}
We have that 
\begin{align*}
    \ket{0} &= c((\omega+\Delta)v_{+}-\lambda v_{-})\\
    &= cd \begin{pmatrix}
        (\omega+\Delta)^2+\lambda^2\\
        0
    \end{pmatrix}
\end{align*}
Therefore we have that $c=d$ and thus we have 
$$\ket{0} = d((\omega+\Delta)v_{+}-\lambda v_{-})$$
Likewise we have that 
$$\ket{1} = d(\lambda v_{+}+(\omega+\Delta)v_{-})$$
Let us denote $v_{\pm} = \ket{\pm}$. We have that the time evolution of 
$\ket{0,t}$ is given by
$$\ket{0,t} = d((\omega+\Delta)e^{-\frac{iE_{+}t}{\hbar}}\ket{+}-
\lambda e^{-\frac{iE_{-}t}{\hbar}}\ket{-})$$
Therefore we have that the probability that the 
system is in state $\ket{1}$ is given by 
\begin{align*}
    |\bra{1}\ket{0,t}|^2 &= (\omega+\Delta)^2\lambda^2 d^4\left| e^{-\frac{iE_{+}t}{\hbar}}
    -e^{-\frac{iE_{-}t}{\hbar}}\right|^2\\
    &= (\omega+\Delta)^2\lambda^2 d^4 (2-2\cos\left(\frac{E_{+}-E_{-}}{\hbar}t\right))\\
    &= 4 (\omega+\Delta)^2\lambda^2 \frac{1}{4\Delta^2(\omega+\Delta)^2} \sin(\Delta t)^2\\
    &= \frac{\lambda^2}{\Delta^2}\sin(\Delta t)^2
\end{align*}
\subsection*{(c)}
We have that the first order correction is:
\begin{align*}
    c_{1}^{(1)}(t) &= \frac{1}{i\hbar}\int_{0}^{t} e^{-2\omega i t'} \hbar\lambda dt\\
    &= \frac{\lambda}{2\omega}\left(e^{-2\omega i t}-1\right)
\end{align*}
Therefore the approximate probability that the system is in state $\ket{1}$ is given by
\begin{align*}
    |c_{0}^{(1)}(t)|^2 &= \frac{\lambda^2}{4\omega^2}\left(e^{-2\omega i t}-1\right)\left(e^{2\omega i t}-1\right)\\
    &= \frac{\lambda^2}{4\omega^2}\left(2-2\cos(2\omega t)\right)\\
    &= \frac{\lambda^2}{\omega^2}\sin^2(\omega t)
\end{align*}
\subsection*{(d)}
If $\omega>>\lambda$ then we have that we can write our transition probability
as 
\begin{align*}
    \frac{\lambda^2}{\Delta^2}\sin(\Delta t)^2 &= \frac{\lambda^2}{\omega^2\left(1+\frac{\lambda^2}{\omega^2}\right)}\sin\left(\omega\sqrt{1+\frac{\lambda^2}{\omega^2}}t\right)
\end{align*}
As $\omega>>\lambda$ we have that $\frac{\lambda}{\omega}$ therefore 
if we perform a taylor expansion of $\frac{\lambda^2}{\Delta^2}\sin(\Delta t)^2$
around $\frac{\lambda}{\omega}=0$ we get that the $0$th order term is
$$\frac{\lambda^2}{\omega^2}\sin^2(\omega t)$$
Which is what we get from perturbation theory.

\end{document}
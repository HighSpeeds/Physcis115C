\documentclass[11pt]{article}
\author{Lawrence Liu}
\usepackage{subcaption}
\usepackage{graphicx}
\usepackage{amsmath,amssymb,stmaryrd}
\usepackage{physics}
\usepackage{pdfpages}
\newcommand{\Laplace}{\mathscr{L}}
\setlength{\parskip}{\baselineskip}%
\setlength{\parindent}{0pt}%
\usepackage{xcolor}
\usepackage{listings}
\definecolor{backcolour}{rgb}{0.95,0.95,0.92}
\usepackage{amssymb}
\lstdefinestyle{mystyle}{
    backgroundcolor=\color{backcolour}}
\lstset{style=mystyle}
\title{Physics 115C HW 3}
\begin{document}
\maketitle
\section*{Problem 1}
\subsection*{(a)}
We have that 
\begin{align*}
    \frac{d}{dt}<xp> &= \frac{i}{\hbar}<[xp,H]>+\left<\frac{\partial xp}{\partial t}\right>\\
    &= \frac{i}{\hbar}<xpH-Hxp>\\
    &= \frac{i}{\hbar}<xpH>-\frac{i}{\hbar}<Hxp>\\
    &= \frac{i}{\hbar}\left<xp\left(\frac{p^2}{2m}+V(x)\right)\right>-\frac{i}{\hbar}\left<\left((\frac{p^2}{2m}+V(x))\right)xp\right>\\
    &= \frac{i}{\hbar}\left<x\frac{p^3}{2m}+xV(x)p+xpV(x)\right>-\frac{i}{\hbar}\left<x\frac{p^3}{2m}+i\hbar\frac{p^2}{m}+V(x)xp\right>\\
    &= \left<\frac{p^2}{m}\right>-\left<\frac{dV(x)}{dx}\right>\\
    &= \left<T\right>-\left<\frac{dV(x)}{dx}\right>
\end{align*}
\subsection*{(b)}
If we have that a state is stationary then we have that 
$$\psi(x,t) = \psi(x)e^{-iEt/\hbar}$$
Thus we have 
\begin{align*}
\frac{\partial}{\partial t} <xp> &= \frac{\partial}{\partial t}\int \psi^*(x,t) xp \psi(x,t) dx\\
&= \frac{\partial}{\partial t}\int \psi^*(x)e^{iEt/\hbar} xp \psi(x)e^{-iEt/\hbar} dx\\
&= \frac{\partial}{\partial t}\int \psi^*(x) xp \psi(x) dx\\
&= 0
\end{align*}
Therefore we have that 
$$0 = 2\left<T\right>-\left<\frac{dV(x)}{dx}\right>$$
$$2\left<T\right> = \left<\frac{dV(x)}{dx}\right>$$
\subsection*{(c)}
We have that 
\begin{align*}
\left<x\frac{dV(x)}{dx}\right> &= \left<m\omega^2x^2\right>\\
&= 2<V>\\
\end{align*}
Therefore we have that 
$$<T> = <V>$$
And thus we have that since 
$$<T>+<V>=E_n$$
$$<T>=<V>=\frac{E_n}{2}$$
\subsection*{(d)}
\begin{align*}
    \frac{d}{dt}\left<r\cdot p\right> &= \frac{i}{\hbar}<[H,rp]>+\left<\frac{\partial rp}{\partial t}\right>\\
    &= \frac{i}{\hbar}<H\cdot r \cdot p-r\cdot p \cdot H>\\
    &= \frac{i}{\hbar}<H\cdot r \cdot p>-\frac{i}{\hbar}<r \cdot p \cdot H>\\
    &= \frac{i}{\hbar}\left<\left(-\frac{\hbar^2}{2m}\nabla^2+V(r)\right)\cdot r\cdot p\right>-\frac{i}{\hbar}\left<r\cdot p\cdot\left(-\frac{\hbar^2}{2m}\nabla^2+V(r)\right)\right>\\
    &= \frac{i}{\hbar}\left<-r\frac{\hbar^2}{2m}\nabla^3-i\hbar\frac{\hbar^2}{2m}\nabla^2+V\cdot r\cdot p\right> - \frac{i}{\hbar}\left<-r\frac{\hbar^2}{2m}\nabla^3-i\hbar\frac{\hbar^2}{2m}\nabla^2+V\cdot r\cdot p+ r\cdot p \cdot V\right>\\
    &= 2\left<T\right>-\left<r\cdot \nabla V\right>\\
\end{align*}
Therefore for a stationary state we have that
$$0 = 2\left<T\right>-\left<r\cdot \nabla V\right>$$
$$2\left<T\right> = \left<r\cdot \nabla V\right>$$
\subsection*{(e)}
We have that for the hydrogen atom:
\begin{align*}
    \left<r\cdot \nabla V\right> &= \left<r\cdot \frac{e^2}{4\pi\epsilon_0r^2}\right>\\
    &= \left<\frac{e^2}{4\pi\epsilon_0r}\right>\\
    &= -\left<V\right>\\
\end{align*}
Therefore we have that
$$2\left<T\right> = -\left<V\right>$$
$$\left<T\right> + \left<V\right> = E_n$$
$$-\left<T\right> = E_n$$
Therefore we have that
$$\left<T\right> = -E_n$$
And:
$$\left<V\right> = 2E_n$$



\section*{Problem 2}
\subsection*{(a)}
We have that 
$$H = -\gamma\frac{\hbar}{2} \begin{bmatrix}
    B_z & B_x\\
    B_x & -B_z
\end{bmatrix}$$
Therefore we can see that the the characteristic equation is
$$\det(H - \lambda I) = \lambda^2 -\left(\gamma\frac{\hbar}{2}\right)^2(B_z^2+B_x^2) = 0$$
Therefore we have that the energy eigenvalues are
$$E = \pm \gamma\frac{\hbar}{2}\sqrt{B_z^2+B_x^2}$$
\subsection*{(b)}
We have that 
\begin{align*}
    \bar{H}&=\bra{\psi_{trial}}H\ket{\psi_{trial}}\\
    &=-\gamma\frac{\hbar}{2}\begin{bmatrix}
        \cos(\phi /2) & \sin(\phi /2)
    \end{bmatrix}
    \begin{bmatrix}
        B_z\cos(\phi /2)+B_x\sin(\phi /2)\\
        B_x\cos(\phi /2)-B_z\sin(\phi /2)
    \end{bmatrix}\\
    &=-\gamma\frac{\hbar}{2}\left(B_z\cos^2(\phi /2)-B_z\sin^2(\phi /2)+2B_x\cos(\phi /2)\sin(\phi /2)\right)\\
\end{align*}
To minimize this we take the derivative with respect to $\phi$ and set it equal to zero
\begin{align*}
    \frac{\partial}{\partial \phi} \bar{H} &= -\gamma\frac{\hbar}{2}\left(-B_z\cos(\phi/2)\sin(\phi/2)-B_z\cos(\phi/2)sin(\phi/2)+B_x\cos^2(\phi/2)-B_x\sin^2(\phi/2)\right)\\
    &=-\gamma\frac{\hbar}{2}\left(B_x\cos(\phi)-B_z\sin(\phi)\right)\\
\end{align*}
Stetting this to zero we get $\phi=\tan^{-1}\left(\frac{B_x}{B_z}\right)$ and therefore 
we get that 
$$\bar{H} = -\gamma\frac{\hbar}{2}\left(\frac{B_z}{\sqrt{1+\left(\frac{B_x}{B_z}\right)^2}}+\frac{B_x\frac{B_x}{B_z}}{\sqrt{1+\left(\frac{B_x}{B_z}\right)^2}}\right)$$
$$\bar{H} = -\gamma\frac{\hbar}{2}\sqrt{B_z^2+B_x^2}$$
\subsection*{(c)}
We recover exactly the same result in part (a). There was a degree of 
freedom which we did not use in part (a) which was the phase of the
eigenstate. We can see that in our trial eigenstate it was all real. 
\section*{Problem 3}
\subsection*{(a)}
We have 
\begin{align*}
    \bra{\psi_{trial}}H\ket{\psi_{trial}} &= \sum_{i=0}^{n} \bra{\psi_{trial}}\ket{\psi_i}\bra{\psi_i}H\ket{\psi_i}\bra{\psi_i}\ket{\psi_{trial}}
\end{align*}
Since 
$$\bra{\psi_{trial}}\ket{\psi_0} = 0$$
We have that 
\begin{align*}
    \bra{\psi_{trial}}H\ket{\psi_{trial}} &= \sum_{i=1}^{n} \bra{\psi_{trial}}\ket{\psi_i}\bra{\psi_i}H\ket{\psi_i}\bra{\psi_i}\ket{\psi_{trial}}\\
    &= \sum_{i=1}^{n} \left|\bra{\psi_{trial}}\ket{\psi_i}\right|^2E_i\\
    &\geq E_1\sum_{i=1}^{n} \left|\bra{\psi_{trial}}\ket{\psi_i}\right|^2\\
    &= E_1
\end{align*}
\subsection*{(b)}
\includegraphics*[scale=0.5]{prob3b.png}
\subsection*{(c)}
We have that for $x_1<\frac{a}{2}$
$$\psi_{trial}(x_1) = Nx_1(\frac{a}{2}-x_1)(a-x_1)$$
Let us defmine $x_2$ as a the location the same distance away from the center 
of the well on the other side of the center. Therefore we have that
$x_2 = a-x_1$. Therefore we have that
$$\psi_{trial}(x_2) = N(a-x_1)(x_1-\frac{a}{2})x_1$$
As we can see $\psi_{trial}(x_1) = -\psi_{trial}(x_2)$ and therefore
the wavefunction is parity odd. \\\\
We have that $\psi_{0}(x) = N'\sin(\pi x/a)$ is parity even around $a/2$
and therefore we have that 
$\psi_0(x)\psi_{trial}(x)$ is parity odd. Therefore we have that the integral of 
it is $0$. Therefore we have that our trial wavefunction is orthogonal to the
ground state wavefunction.
\subsection*{(d)}
We have that our ground state trial wavefunction is parity even,
thus the multiple of it with our first excited state is parity odd.
Therefore we have that the integral of it is $0$. Therefore we have that our trial wavefunction is orthogonal to the
first excited state wavefunction.
\subsection*{(e)}
$$\int_{0}^{a} (x(a/2-x)(a-x))^2 dx = \frac{a^7}{840}$$
Therefore $$N=\sqrt{\frac{840}{a^7}}$$
\subsection*{(f)}
We have that 
$$-\frac{\hbar^2}{2m}\frac{d^2}{dx^2}\psi_{trial}(x) = -\frac{3\hbar^2}{m}\left(x-\frac{a}{2}\right)$$
Therefore we have that 
$$\bar{H} = \frac{\hbar^2 a^5}{40m} \frac{840}{a^7}$$
$$\bar{H} = \frac{21\hbar^2}{ma^2}$$
$$\bar{H} = \frac{42\hbar^2}{2ma^2}$$
This is not that far off from the actual first excited state 
energy of $\frac{(2\pi)^2\hbar^2}{2ma^2}=\frac{39.478\hbar^2}{2ma^2}$. With a difference of only 
around $6.3\%$.


% Therefore we can see that we 
\end{document}
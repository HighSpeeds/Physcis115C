\documentclass[11pt]{article}
\author{Lawrence Liu}
\usepackage{subcaption}
\usepackage{graphicx}
\usepackage{amsmath,amssymb,stmaryrd}
\usepackage{physics}
\usepackage{pdfpages}
\newcommand{\Laplace}{\mathscr{L}}
\setlength{\parskip}{\baselineskip}%
\setlength{\parindent}{0pt}%
\usepackage{xcolor}
\usepackage{listings}
\definecolor{backcolour}{rgb}{0.95,0.95,0.92}
\usepackage{amssymb}
\lstdefinestyle{mystyle}{
    backgroundcolor=\color{backcolour}}
\lstset{style=mystyle}
\title{Physics 115C Hw 2}
\begin{document}
\maketitle
\section*{Problem 1}
\subsection*{(a)}
We have that the first order corrections to the energy is given by 
$$
E_{n}^{1}=\bra{\psi_n}H'\ket{\psi_n}
$$
Therefore in our case we have that 
$$
    E_{n}^{1}=\alpha\bra{\psi_n}\delta(x-\frac{L}{2})\ket{\psi_n}
$$
Therefore we get that 
\begin{align*}
    E_{n}^{1}&=\alpha\int
    \psi_n^*(x) \delta(x-\frac{L}{2})\psi_n(x)dx\\
    &=\alpha |\psi(\frac{L}{2})|^2 \\
    &= \alpha \frac{2}{L} \sin^2\left(\frac{n\pi}{2}\right)\\
    &=\begin{cases}
    \alpha \frac{2}{L} & \text{if $n$ is odd}\\
    0 & \text{if $n$ is even}
    \end{cases}
\end{align*}
\subsection*{(b)}
\includegraphics*[width=0.5\textwidth]{wavefunction.png}\\
As we can see the states that do not exhibit a energy 
shift of the first order, are states where the probability
density is zero at the location of the perubation. Thus the 
particle will not feel the effect of the perubation.
\subsection*{(c)}
Likewise we can see that the states that do exhibit a energy
shift are states where the probability density is not zero at the
location of the perubation. Thus the particle will feel the effect
of the perubation.
\subsection*{(d)}
The delta function acts to effectively "sample" the wavefunction for the 
first order correction in the energy. Thus for the maximum first order 
energy corrections, we would want to place the delta function 
at the location where the probability density is the largest. Therefore for the 
$n=2$ state this will be $\delta(x-\frac{L}{4}),
\delta(x-\frac{3L}{4})$. For the $n=4$ state this would be 
$\delta(x-\frac{L}{8}),\delta(x-\frac{3L}{8}),\delta(x-\frac{5L}{8}),
\delta(x-\frac{7L}{8})$. And for 
the $n=6$ state this would be
$\delta(x-\frac{1}{12}L),\delta(x-\frac{3}{12}L),\delta(x-\frac{5}{12}L),
\delta(x-\frac{7}{12}L),\delta(x-\frac{9}{12}L),\delta(x-\frac{11}{12}L)$.
\subsection*{(e)}
Since it samples the wavefunction, as we move the delta function, the resulting 
first order correction will just be the amplitude of the wavefunction at the
delta function:\\
\includegraphics*[width=0.5\textwidth]{fig1.png}
\section*{Problem 2}
\subsection*{(a)}
$H^0=\omega S_z$, thereofre the eigenvalues are $+\omega$ and $-\omega$, and the 
eigenstates are $\ket{\uparrow_z}=[1,0]^T$ and $\ket{\downarrow_z}=[0,1]^T$.
\subsection*{(b)}
We have that the characteristic equation is given by
$$x^2-\omega^2-(\lambda\Omega)^2=0$$
Therefore the egienvalues are $\pm\sqrt{\omega^2+(\lambda\Omega)^2}$.
\subsection*{(c)}
We have that 
$$H\ket{\psi_1}=\left(\omega\cos\left(\frac{\phi}{2}\right)
+\lambda\Omega\sin(\frac{\phi}{2})\right)\ket{0}+\left(\lambda\Omega\cos\left(\frac{\phi}{2}\right)
-\omega\sin(\frac{\phi}{2})\right)\ket{1}$$
In order $\ket{\psi_1}$ to be an eigenstate we must have:
$$
H\ket{\psi_1} = \sqrt{\omega^2+(\lambda\Omega)^2}\ket{\psi_1}
$$
ie we need:
\begin{align*}
    \omega\cos\left(\frac{\phi}{2}\right)
+\lambda\Omega\sin\left(\frac{\phi}{2}\right)
& = \sqrt{\omega^2+(\lambda\Omega)^2} \cos\left(\frac{\phi}{2}\right)
\\
\lambda\Omega\cos\left(\frac{\phi}{2}\right)
-\omega\sin\left(\frac{\phi}{2}\right) & = \sqrt{\omega^2+(\lambda\Omega)^2} \sin\left(\frac{\phi}{2}\right)
\end{align*}
We have that
\begin{align*}
    \omega\cos\left(\frac{\phi}{2}\right)
+\lambda\Omega\sin\left(\frac{\phi}{2}\right)
& = \sqrt{\omega^2+(\lambda\Omega)^2} \cos\left(\frac{\phi}{2}\right)
\\
\cos(\phi)\cos\left(\frac{\phi}{2}\right)+
\sin(\phi)\sin\left(\frac{\phi}{2}\right) &= \cos\left(\frac{\phi}{2}\right)\\
\sin^2(\phi)\sin^2\left(\frac{\phi}{2}\right) &= (1-\cos(\phi))^2\cos\left(\frac{\phi}{2}\right)\\
\sin^2(\phi)\frac{1-\cos(\phi)}{2} &= (1-\cos(\phi))^2\frac{1+\cos(\phi)}{2}\\
(1-\cos^2(\phi))(1-\cos(\phi))&=(1-\cos(\phi))^2(1+\cos(\phi))\\
(1-\cos(\phi))^2(1+\cos(\phi))&=(1-\cos(\phi))^2(1+\cos(\phi))
\end{align*}
Likewise we have that 
\begin{align*}
    \lambda\Omega\cos\left(\frac{\phi}{2}\right)
    -\omega\sin\left(\frac{\phi}{2}\right) & = \sqrt{\omega^2+(\lambda\Omega)^2} \sin\left(\frac{\phi}{2}\right)\\
    \sin(\phi)\cos\left(\frac{\phi}{2}\right)-\cos(\phi)\sin\left(
    \frac{\phi}{2}\right) &= \sin\left(\frac{\phi}{2}\right)\\
    \sin^2(\phi)\cos^2\left(\frac{\phi}{2}\right) &= (1+\cos(\phi))^2\sin^2\left(\frac{\phi}{2}\right)\\
    \sin^2(\phi)(1+\cos(\phi)) &= (1+\cos(\phi))^2(1-\cos(\phi))\\
    (1-\cos^2(\phi))(1+\cos(\phi)) &= (1+\cos(\phi))^2(1-\cos(\phi))\\
    (1+\cos(\phi))^2(1-\cos(\phi)) &= (1+\cos(\phi))^2(1-\cos(\phi))
\end{align*}
Therefore as we can see $\ket{\psi_1}$ is the eigenstate 
for the energy eigenvalue $\sqrt{\omega^2+(\lambda\Omega)^2}$.\\\\
We can repeat a similar process for $\ket{\psi_2}$, we want to have 
$$
H\ket{\psi_2} = -\sqrt{\omega^2+(\lambda\Omega)^2}\ket{\psi_2}
$$
ie we need:
\begin{align*}
    -\omega\sin\left(\frac{\phi}{2}\right)+\lambda\Omega\cos\left(
        \frac{\phi}{2}\right) &= \sqrt{\omega^2+(\lambda\Omega)^2}\sin\left(\frac{\phi}{2}\right)\\
    -\lambda\Omega\sin\left(\frac{\phi}{2}\right)-\omega\cos\left(\frac{\phi}{2}\right)
    &= -\sqrt{\omega^2+(\lambda\Omega)^2}\cos\left(\frac{\phi}{2}\right)
\end{align*}
For the first equation we have that 
\begin{align*}
    -\omega\sin\left(\frac{\phi}{2}\right)+\lambda\Omega\cos\left(
        \frac{\phi}{2}\right) &= \sqrt{\omega^2+(\lambda\Omega)^2}\sin\left(\frac{\phi}{2}\right)\\
    -\cos(\phi)\sin\left(\frac{\phi}{2}\right)+\sin(\phi)\cos\left(\frac{\phi}{2}\right) &= \sin\left(\frac{\phi}{2}\right)\\
    % -\sin(\phi)\cos\left(\frac{\phi}{2}\right) &= (1-\cos(\phi))\sin\left(\frac{\phi}{2}\right)
\end{align*}
We already proved equality holds for this when we proved that the $\ket{1}$ components 
of $H\ket{\psi_1}$ and $\sqrt{\omega^2+(\lambda\Omega)^2}\ket{\psi_1}$ are equal. For the second equation we have 
\begin{align*}
    -\lambda\Omega\sin\left(\frac{\phi}{2}\right)-\omega\cos\left(\frac{\phi}{2}\right)
    &= -\sqrt{\omega^2+(\lambda\Omega)^2}\cos\left(\frac{\phi}{2}\right)\\
    \cos(\phi)\cos\left(\frac{\phi}{2}\right)+
\sin(\phi)\sin\left(\frac{\phi}{2}\right) &= \cos\left(\frac{\phi}{2}\right)
\end{align*}
We already proved equality holds for this when we proved that the $\ket{0}$ components 
of $H\ket{\psi_1}$ and $\sqrt{\omega^2+(\lambda\Omega)^2}\ket{\psi_1}$ are equal. Thus $\ket{\psi_2}$ is the eigenstate 
for the energy eigenvalue $-\sqrt{\omega^2+(\lambda\Omega)^2}$.
\subsection*{(d)}

\end{document}

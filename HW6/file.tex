\documentclass[11pt]{article}
\author{Lawrence Liu}
\usepackage{subcaption}
\usepackage{graphicx}
\usepackage{amsmath,amssymb,stmaryrd}
\usepackage{physics}
\usepackage{pdfpages}
\newcommand{\Laplace}{\mathscr{L}}
\setlength{\parskip}{\baselineskip}%
\setlength{\parindent}{0pt}%
\usepackage{xcolor}
\usepackage{listings}
\definecolor{backcolour}{rgb}{0.95,0.95,0.92}
\usepackage{amssymb}
\lstdefinestyle{mystyle}{
    backgroundcolor=\color{backcolour}}
\lstset{style=mystyle}
\title{Physics 115C HW 5}
\begin{document}
\maketitle
\section*{Problem 1}
\subsection*{(a)}
We have that 
$$\Delta=\frac{\Omega}{2}$$
Therefore we must have that 
$$\frac{\lambda^2}{\Delta^2}=4\frac{\lambda^2}{\Omega^2}=\frac{\gamma^2}{\hbar^2\Omega^2}$$
Therefore we have that 
$$\lambda = \frac{\gamma}{2\hbar}$$
Since 
$$\Delta^2=\lambda^2+\omega^2$$
We have that 
$$\omega^2=\frac{\gamma^2}{4\hbar^2}-\frac{\Omega^2}{4}$$
Since $\Omega^2 = \delta\omega^2+\frac{\gamma^2}{\hbar^2}$ we have:
$$\omega^2=-\frac{\delta\omega^2}{4}$$
Therefore we have that
$$\omega = i\frac{\delta\omega}{2}$$
\subsection*{(b)}
Therefore we have that 
$$H_0=\begin{pmatrix}
    2\hbar i\frac{\delta\omega}{2} & 0\\
    0 & 0
\end{pmatrix}$$
And 
$$H_1=\begin{pmatrix}
    0 & \frac{\gamma}{2}\\
    \frac{\gamma}{2} & 0
\end{pmatrix}$$
Therefore we have that 
$$H_0+H_1 = \begin{pmatrix}
    2\hbar i\frac{\delta\omega}{2}& \frac{\gamma}{2}\\
    \frac{\gamma}{2} & 0
\end{pmatrix}$$
We know that the eigenvalues in terms of $\lambda$, $\omega$ and $\Delta$
are $\hbar(\omega\pm\Delta)$, therefore we have that the energy eigenvalues are 
$$E_{\pm} = \hbar \left(i\frac{\delta\omega}{2}\pm \frac{\Omega}{2}\right)$$
And we have that the eigenvectors $v_{+}= d\begin{pmatrix}
    \omega+\Delta\\
    \lambda
\end{pmatrix}$ and $v_{-}= d\begin{pmatrix}
    -\lambda\\
    \omega+\Delta
\end{pmatrix}$ where $d$ is a normalization constant. Therefore we have that
$$v_{+}=d\begin{pmatrix}
    i\frac{\delta\omega}{2}+\frac{\Omega}{2}\\
    \frac{\gamma}{2\hbar}
\end{pmatrix}$$
And
$$v_{-}=d\begin{pmatrix}
    -\frac{\gamma}{2\hbar}\\
    i\frac{\delta\omega}{2}+\frac{\Omega}{2}
\end{pmatrix}$$
We have that $d^{-2} = \Omega$.
\subsection*{(c)}
\includegraphics*[width=0.7\columnwidth]{fig1.png}\\
This is a lorentzian curve.
\section*{Problem 2}
We have that the probability of a transition is given by
$$P_{if}=\left|\frac{i}{\hbar}\int_{0}^{t} e|E_0|e^{-\gamma t} \bra{f}z\ket{i}dt\right|^2$$
We have that
$$\bra{211}z\ket{100}=0$$
$$\bra{21-1}z\ket{210}=0$$
From 6.3b. Also from applying what we have from 6.3e and changing 
$L_z$ to $L_y$ and $y$ to $z$ we get:
$$\bra{200}z\ket{100}=0$$
And thus we just need to solve for 
\begin{align*}
    \bra{210}z\ket{100}&=\int_{0}^{\infty}\int_{0}^{2\pi}\int_{0}^{\pi} r^3\sin(\theta)\cos(\theta)
    \psi_{210}^{*}(r,\theta,\phi)\psi_{210}^{*}(r,\theta,\phi)d\theta d\phi dr\\
    &= 2\pi\int_{0}^{\infty}Y_{210} Y_{100} r^3\int_{0}^{\pi}\sin(\theta)\cos^2(\theta)dr\\
    &= \frac{4\pi}{3}\int_{0}^{\infty}Y_{210} Y_{100} r^3dr\\
    &= \frac{4\pi}{3}\frac{1}{4\sqrt{2\pi}}\frac{1}{\sqrt{\pi}a_0^{4}}\int_{0}^{\infty} r^4 e^{-3r/2a_0}dr\\
    &= \frac{2^\frac{15}{2}a_0}{3^5}
\end{align*}
Thereforer we have that the probability of transistion is given by 
$$P_{if}=\left|\frac{i}{\hbar}\frac{2^\frac{15}{2}a_0}{3^5} \int_{0}^{t} e|E_0|e^{-\gamma t}e^{\frac{it}{\hbar}E_f-E_i}dt\right|^2$$
Thus the limit as $t\rightarrow\infty$ is:
$$P_{if}=\left|\frac{i}{\hbar}\frac{2^\frac{15}{2}a_0}{3^5} \frac{e|E_0|}{-\gamma+\frac{i(E_f-E_i)}{\hbar}}\right|^2$$
$$P_{if}=\frac{2^{15}a_0^2}{3^{10}\hbar^2} \frac{e^2|E_0|^2}{\gamma^2+\frac{(E_f-E_i)^2}{\hbar^2}}$$
We have that $E_2-E_1 = \frac{3e^2}{8a_0}$ thus we get:
$$P_{if}=\boxed{\frac{2^{15}a_0^2}{3^{10}\hbar^2} \frac{e^2|E_0|^2}{\gamma^2+\frac{9e^4}{64a_0^2\hbar^2}}}$$
\section*{Problem 3}
\subsection*{(a)}
We have that 
\begin{align*}
    [L_z,x] &= [xp_y-yp_x,x]\\
    &= [xp_y,x]-[yp_x,x]\\
    &= x[p_y,x]+[x,p_y]x-[y,x]p_x-y[p_x,x]\\
    &= -y[p_x,x]\\
    &= yi\hbar
\end{align*}
Likewise 
\begin{align*}
    [L_z,y] &= [xp_y-yp_x,y]\\
    &= [xp_y,y]-[yp_x,y]\\
    &= x[p_y,y]+[x,p_y]y-[y,x]p_x-y[p_x,y]\\
    &= x[p_y,y]-[y,x]p_x\\
    &= -xi\hbar
\end{align*}
and 
\begin{align*}
    [L_z,z] &= [xp_y-yp_x,z]\\
    &= [xp_y,z]-[yp_x,z]\\
    &= x[p_y,z]+[x,p_y]z-[y,x]p_x-y[p_x,z]\\
    &= x[p_y,z]-[y,x]p_x\\
    &= 0
\end{align*}
\subsection*{(b)}
We have that 
\begin{align*}
    \bra{n'l'm'}[L_z,z]\ket{nlm} &= \bra{n'l'm'}0\ket{nlm}\\
    \bra{n'l'm'}L_zz-zL_z\ket{nlm} &= 0\\
    \bra{n'l'm'}(m'-m)z\ket{nlm} &= 0\\
    (m'-m)\bra{n'l'm'}z\ket{nlm} &= 0
\end{align*}
Thus we can see that unless $m'=m$ the matrix element must be zero, thus if 
light is polarized in the $z$ direction, it cannot change the $m$ quantum number.
Thus we cannot go from $\ket{100}\to\ket{211}$ with light polarized in the $z$ direction.
\subsection*{(c)}
We have that
\begin{align*}
    \bra{n'l'm'}[L_z,x]\ket{nlm} &= \bra{n'l'm'}i\hbar y\ket{nlm}\\
    \bra{n'l'm'}L_zx-xL_z\ket{nlm} &= i\hbar\bra{n'l'm'}y\ket{nlm}\\
    \hbar \bra{n'l'm'}(m'-m)x\ket{nlm} &= i\hbar\bra{n'l'm'}y\ket{nlm}\\
    (m'-m)\bra{n'l'm'}x\ket{nlm} &= i\bra{n'l'm'}y\ket{nlm}
\end{align*}
\subsection*{(d)}
We have that:
\begin{align*}
    [L_z,x\pm i y] &= [L_z,x]\pm i[L_z,y]\\
    &= i\hbar y\pm \hbar x\\
\end{align*}
\begin{align*}
    \bra{n'l'm'}[L_z,x\pm i y]\ket{nlm} &=  \bra{n'l'm'}i\hbar y\pm \hbar x\ket{nlm}\\
    (m'-m)\bra{n'l'm'}x\pm iy\ket{nlm} &= \pm\bra{n'l'm'}x\pm iy\ket{nlm}\\
    (m'-m\mp 1)\bra{n'l'm'}x\pm iy\ket{nlm} &= 0
\end{align*}
Thus we can see that the only possible nonzero matrix elements are when $m'=m\pm 1$. This 
reminds us of the raising and lower operators.
\subsection*{(e)}
We have that 
\begin{align*}
    \bra{n'l'm'}[L_z,y]\ket{nlm} &= -\bra{n'l'm'}i\hbar x\ket{nlm}\\
    (m'-m) \bra{n'l'm'}y\ket{nlm} &= -i\bra{n'l'm'}x\ket{nlm}\\
    (m'-m)^2 \bra{n'l'm'}x\ket{nlm} &= \bra{n'l'm'}x\ket{nlm}\\
    \left((m'-m)^2-1\right)\bra{n'l'm'}x\ket{nlm} &= 0
\end{align*}
Thus we can see that $\bra{n'l'm'}x\ket{nlm}= \bra{n'l'm'}y\ket{nlm} = 0$
unless $(m'-m)^2=1$. This means that for both the $bra{n'l'm'}x\ket{nlm}$ and 
$\bra{n'l'm'}y\ket{nlm}$ matricies will have 0 values beyond the diagonal above a
and below the main diagonal. 
\end{document}

\documentclass[11pt]{article}
\author{Lawrence Liu}
\usepackage{subcaption}
\usepackage{graphicx}
\usepackage{amsmath,amssymb,stmaryrd}
\usepackage{physics}
\usepackage{pdfpages}
\newcommand{\Laplace}{\mathscr{L}}
\setlength{\parskip}{\baselineskip}%
\setlength{\parindent}{0pt}%
\usepackage{xcolor}
\usepackage{listings}
\definecolor{backcolour}{rgb}{0.95,0.95,0.92}
\usepackage{amssymb}
\lstdefinestyle{mystyle}{
    backgroundcolor=\color{backcolour}}
\lstset{style=mystyle}
\title{Physics 115C HW 3}
\begin{document}
\maketitle
\section*{Problem 1}
\subsection*{(a)}
We have that:
\begin{align*}
    c_{f}^{(1)} &= \frac{1}{i\hbar}\int_{0}^{t} e^{i\omega_{fi}t'}\bra{f}V\ket{i}dt'\\
    &= \frac{1}{i\hbar}\bra{f}V\ket{i}\int_{0}^{t} e^{i\omega_{fi}t'}dt'\\
    &= \frac{1}{i\hbar}\bra{f}V\ket{i}\frac{1}{i\omega_{fi}}\left(e^{i\omega_{fi}t}-1\right)\\
    &= -\frac{V_{fi}}{\hbar\omega_{fi}}\left(e^{i\omega_{fi}t}-1\right)\\
    \left|c_{f}^{(1)}\right|^2 &= \frac{V_{fi}^2}{\hbar^2\omega_{fi}^2}\left(2-2\cos\left(\omega_{fi}t\right)\right)\\
    &= \frac{|V_{fi}|^2}{\hbar^2\omega_{fi}^2}\left(4\sin^2\left(\frac{\omega_{fi}t}{2}\right)\right)\\
\end{align*}
\subsection*{(b)}
Therefore we have that the probability of transition to a group of final states is given by:
$$P(t) = \sum_{f} \frac{4|V_{fi}|^2}{\hbar^2\omega_{fi}^2}\left(\sin^2\left(\frac{\omega_{fi}t}{2}\right)\right)$$
\subsection*{(c)}
If we assume that the states for a continuum of final states, then we have that:
$$P(t) = \frac{1}{\hbar^2}\int |V_{fi}|^2\left(\frac{\sin\left(\frac{(E_{f}-E_i)t}{2\hbar}\right)}{\frac{E_f-E_i}{2\hbar}}\right)^2\rho(E_f)dE_f$$
\subsection*{(d)}
These approximations allow us to bring $|V_{fi}|^2$ and $\rho(E_f)$ outside of the integral, and we have that:
$$P(t) = \frac{1}{\hbar^2}|V_{fi}|^2\rho(E_f)\int \left(\frac{\sin\left(\frac{(E_{f}-E_i)t}{2\hbar}\right)}{\frac{E_f-E_i}{2\hbar}}\right)^2dE_f$$
We can make this approximation if we take the long time limit since the peak of the sinc function will be very narrow, with the 
width scaling with $\frac{1}{t}$, and thus when $t>>\frac{\hbar}{(E_{f}-E_i)}$ then we will have that the width $\frac{1}{t}<<\frac{E_f-E_i}{\hbar}$,
and thus we can assume that the density of states is roughly constant over the width of the sinc function.
\subsection*{(e)}
$$\int \left(\frac{\sin\left(\frac{(E_{f}-E_i)t}{2\hbar}\right)}{\frac{E_f-E_i}{2\hbar}}\right)^2dE_f = 2\hbar t \int_{-\infty}^{\infty} \left(\frac{\sin(x)}{x}\right)^2dx=2\hbar t\pi$$
Therefore we have:
$$P(t) = \frac{2\pi}{\hbar}|V_{fi}|^2\rho(E_f)t$$
And thus the transision rate is given by:
$$\Gamma = \frac{2\pi}{\hbar}|V_{fi}|^2\rho(E_f)$$
\section*{Problem 2}
\subsection*{(a)}
The ground state of a harmonic oscillator is given by:
$$\psi_0(x) = \left(\frac{m\omega}{\pi\hbar}\right)^{1/4}e^{-\frac{m\omega}{2\hbar}x^2}$$
\subsection*{(b)}
We have that since the final wavefunction is given by 
$$\psi_f(x) = Ce^{ikx}$$
Since the periodic boundary conditions require that $\psi_f(x) = \psi_f(x+L)$, we have that:
$$k = \frac{2\pi n}{L}$$
for $n = 0, \pm 1, \pm 2, ...$.
Therefore we have 
$$\psi_f(x) = \frac{1}{\sqrt{L}}e^{i\frac{2\pi n}{L}x}$$
\subsection*{(c)}
We have that from fermi's golden rule, the transition probability is given by:
$$P_{0\rightarrow f} = \frac{2\pi}{\hbar}|V_{fi}|^2\rho(E_f)$$
\subsection*{(d)}
And we have that the matrix element is 
$$\bra{f} V \ket{i} = E_0e^{-i\omega t}\left(\frac{m\omega}{\pi\hbar}\right)^{1/4}\frac{1}{\sqrt{L}}\int_{-\infty}^{\infty} e^{-i\frac{2\pi n}{L}x}xe^{-\frac{m\omega}{2\hbar}x^2}dx$$
To evaluate the integral let $\omega' = \frac{2\pi n}{L}$ then we notice that the 
integral is effectively the fourier transform of $xe^{-\frac{m\omega}{2\hbar}x^2}$ at $\omega'$, and we have that:
$$e^{-\frac{m\omega}{2\hbar}x^2} \to \frac{1}{\sqrt{\frac{m\omega}{\hbar}}}e^{\frac{-\omega'^2}{2\frac{m\omega}{\hbar}}}$$
$$xe^{-\frac{m\omega}{2\hbar}x^2} \to -i\frac{1}{\sqrt{\frac{m\omega}{\hbar}}}\frac{\omega'}{\frac{m\omega}{\hbar}}e^{\frac{-\omega'^2}{2\frac{m\omega}{\hbar}}}$$
Therefore we have that:
$$\int_{-\infty}^{\infty} e^{-i\frac{2\pi n}{L}x}xe^{-\frac{m\omega}{2\hbar}x^2}dx = -i\sqrt{2\pi}\frac{1}{\sqrt{\frac{m\omega}{\hbar}}}\frac{2\pi n \hbar}{Lm\omega}e^{\frac{-2\hbar \pi^2 n^2}{mL\omega}}$$
And thus we have the matrix element:
$$\bra{f} V \ket{i} =-iE_0e^{-i\omega t} \sqrt{2\pi}\left(\frac{m\omega}{\pi\hbar}\right)^{1/4}\frac{1}{\sqrt{L}}\frac{1}{\sqrt{\frac{m\omega}{\hbar}}}\frac{2\pi n \hbar}{Lm\omega}e^{\frac{-2\hbar \pi^2 n^2}{mL\omega}}$$
\subsection*{(e)}
We have that 
$$E = \frac{\hbar^2k^2}{2m} = \frac{\hbar^2}{2m}\left(\frac{2\pi n}{L}\right)^2$$
Therefore we have that:
$$dE = \frac{\hbar^2}{m}\frac{2\pi^2 n}{L^2}dn$$
Therefore we have that 
$$\rho(E) = \frac{L^2}{2\pi^2 n}\frac{m}{\hbar^2}$$
\subsection*{(f)}
$$\Gamma = \frac{2\pi}{\hbar} \frac{L^2}{2\pi^2 n}\frac{m}{\hbar^2}\left(\sqrt{2\pi}E_0\left(\frac{m\omega}{\pi\hbar}\right)^{1/4}\frac{1}{\sqrt{L}}\frac{1}{\sqrt{\frac{m\omega}{\hbar}}}\frac{2\pi n \hbar}{Lm\omega}e^{\frac{-2\hbar \pi^2 n^2}{mL\omega}}\right)^2$$
$$\Gamma = \frac{8\pi^2n}{m\omega^3L}\left(\frac{m\omega}{\pi\hbar}\right)^{\frac{1}{2}}e^{\frac{-2\hbar \pi^2 n^2}{mL\omega}}$$
\section*{Problem 3}
\subsection*{(a)}
We have that applying the fermi golden rule the absorption rate is given by:
$$\Gamma = \frac{2\pi}{\hbar}|U_{fi}|^2\rho(E_f)$$
We also have that since we approximated the peak to be very narrow, we would get that $E_f = E_f-E_i-\hbar\omega$, and thus we have that:
$$\Gamma = \frac{2\pi}{\hbar}|U_{fi}|^2\rho(E_i+\hbar\omega\delta)$$
Likewise we have the emission rate is given by:
$$\Gamma = \frac{2\pi}{\hbar}|U_{fi}|^2\rho(E_i-\hbar\omega\delta)$$
Since we would have that the only possible emission energy is $E_f = E_i-\hbar\omega$, we notice that this satistifies
the energy conservation law. Since the final energies is the same as the initial energy plus or minus the energy of the photon, for the 
absorption and emission respectively.
\subsection*{(b)}
\includegraphics*[scale=0.1]{fig1.png}\\
\subsection*{(c)}
We have that the initial wavefunction would be:
$$\psi_i(x) = \frac{1}{\sqrt{V}}$$
And the final wavefunction would be:
$$\psi_f(x) = \frac{1}{\sqrt{V}}e^{i\mathbf{k}\cdot\mathbf{r}}$$
Therefore we have that the matrix element is given by:
$$\bra{f}\frac{\hbar^2\lambda}{2m}\delta(\mathbf{r})\ket{i} = \frac{\hbar^2\lambda}{2m}\frac{1}{V}\int e^{-i\mathbf{k}\cdot\mathbf{r}}\delta(\mathbf{r})d^4r$$
$$\bra{f}\frac{\hbar^2\lambda}{2m}\delta(\mathbf{r})\ket{i} = \frac{\hbar^2\lambda}{2m}$$
We have that the energy of a given state is given by 
$$E = \frac{\hbar^2}{2m} \mathbf{k}\cdot \mathbf{k}$$
Or in terms of $\mathbf{n} = \frac{L}{2\pi} \mathbf{k}$ we have that:
$$E = \frac{\hbar^2}{2m} \frac{4\pi^2}{L^2} \mathbf{n}\cdot \mathbf{n}$$
We can write $\mathbf{n}\cdot \mathbf{n}$ as effectively a "radius" $r$. Therefore we have that:
$$E = \frac{\hbar^2}{2m} \frac{4\pi^2}{L^2} r^2$$
Thus we have that 
$$r = \frac{L}{2\pi}\sqrt{\frac{2mE}{\hbar^2}}$$
We also have that for a 4d hypersphere with radius r', the surface volumn is given by:
$$V = 2\pi^2r'^3$$
Therefore we have that the density of states is given by:
$$\rho(E_f) = 2\pi^2 \left(\frac{L}{2\pi}\sqrt{\frac{2mE}{\hbar^2}}\right)^3$$
Thus from fermi's golden rule we have that:
$$\Gamma = \frac{4\pi^3}{\hbar}\left(\frac{\hbar^2\lambda}{2m}\right)^2 \left(\frac{L}{2\pi}\sqrt{\frac{2mE}{\hbar^2}}\right)^3$$
\end{document}